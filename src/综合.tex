定理 1: 最小覆盖数 $=$ 最大匹配数\\
定理 2: 最大独立集 S 与 最小覆盖集 T 互补。\\
算法:
\begin{enumerate}
\item[1.] 做最大匹配 , 没有匹配的空闲点 $\in S$
\item[2.] 如果 $u\in S$ 那么 u 的临点必然属于 T
\item[3.] 如果一对匹配的点中有一个属于 T 那么另外一个属于 S
\item[4.] 还不能确定的 , 把左子图的放入 S, 右子图放入 T
\end{enumerate}
算法结束
\\
\\
上下界无源汇可行流 : 不用添 $T-> S$, 判断是否流量平衡\\
上下界有源汇可行流 : 添 $T\rightarrow S$( 下界 $0$, 上界 $\infty$), 判断是否流量平衡\\
上下界最小流 : 不添 $T\rightarrow S$ 先流一遍 , 再添 $T\rightarrow S$( 下界 $0$, 上界 $\infty$) 在残图上流一遍 , 答案为 $S\rightarrow T$ 的流量值\\
上下界最大流 : 添 $T\rightarrow S$( 下界 $0$, 上界 $\infty$) 流一遍,再在残图上流一遍S到T的最大流,答案为前者的 $S\rightarrow T$ 的值 $+$ 残图中 $S\rightarrow T$ 的最大流\\
\\
Stirling 公式\ 
$
    n!=\sqrt{2\pi n}(\frac{n}{e})^n
$
\\
\\
Stirling 数 \\
第一类 :n 个元素的项目分作 k 个环排列的方法数目\\
$
    s(n, k) = (-1)^{n+k}|s(n, k)|
    $\\
    $
    |s(n, 0)|=0, |s(1, 1)|=1,
    $\\
    $
    |s(n, k)|=|s(n-1, k-1)|+(n-1)*|s(n-1, k)|
$
\\
第二类 :n 个元素的集定义 k 个等价类的方法数\\
$
    S(n,1)=S(n,n)=1,
    S(n,k)=S(n-1,k-1)+k*S(n-1,k)
$
